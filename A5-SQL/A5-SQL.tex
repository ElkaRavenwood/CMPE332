\documentclass[letterpaper]{article}
\usepackage{tikz}
\usepackage{float}
\usepackage[margin=1in]{geometry}
\usepackage{hyperref}
\usepackage{listings}
\definecolor{vgreen}{RGB}{104,180,104}
\definecolor{vblue}{RGB}{49,49,255}
\lstset{
    language=SQL,
    % wrap text
    breaklines=true, 
    % other styling
    basicstyle=\small\ttfamily,
    keywordstyle=\color{vblue},
    identifierstyle=\color{black},
    commentstyle=\color{vgreen},
    tabsize=2
}

\title{
    ELEC 332 Assignment 5 \\
    \begin{large}
        SQL
    \end{large}
}
\author{Jamie Won | 20113217 | \href{mailto:jamie.won@queensu.ca}{jamie.won@queensu.ca}}

\begin{document}

\maketitle
\tableofcontents

\cleardoublepage

\section{Query 1}

    \textbf{Show the names of all instructors in the Comp. Sci. department who earn more than \$70,000.}
    \begin{lstlisting}
    SELECT name FROM instructor WHERE dept_name="Comp. Sci." AND salary > 70000


    name	
    Katz	
    Brandt	
    \end{lstlisting}
        
\section{Query 2}

    \textbf{List each student id along with the course ids that they take each year.}
    \begin{lstlisting}
       
    SELECT ID, course_id FROM takes
    
    
    ID	    course_id	
    00128	CS-101	
    00128	CS-347	
    12345	CS-101	
    12345	CS-190	
    12345	CS-315	
    12345	CS-347	
    19991	HIS-351	
    23121	FIN-201	
    44553	PHY-101	
    45678	CS-101	
    45678	CS-101	
    45678	CS-319	
    54321	CS-101	
    54321	CS-190	
    55739	MU-199	
    76543	CS-101	
    76543	CS-319	
    76653	EE-181	
    98765	CS-101	
    98765	CS-315	
    98988	BIO-101	
    98988	BIO-301	
       
    \end{lstlisting}
        
\section{Query 3}

    \textbf{Which time slots have classes on Mondays?  Show the output as a single column labeled MondaySlots.  Be sure to check that your results are correct!}
    \begin{lstlisting}
    SELECT time_slot_id AS MondaySlots FROM time_slot WHERE day="M"


    MondaySlots	
    A	
    B	
    C	
    D	
    G	
    \end{lstlisting}
        
\section{Query 4}

    \textbf{Make a list of student names and their course ids.  Order the results alphabetically by student name.}
    \begin{lstlisting}
    SELECT name, course_id FROM student, takes WHERE student.ID=takes.ID ORDER BY name


    name   	    course_id	
    Aoi	        EE-181	
    Bourikas	CS-101	
    Bourikas	CS-315	
    Brandt      HIS-351	
    Brown       CS-101	
    Brown       CS-319	
    Chavez      FIN-201	
    Levy        CS-101	
    Levy        CS-101	
    Levy        CS-319	
    Peltier     PHY-101	
    Sanchez     MU-199	
    Shankar     CS-101	
    Shankar     CS-190	
    Shankar     CS-315	
    Shankar     CS-347	
    Tanaka      BIO-101	
    Tanaka      BIO-301	
    Williams	CS-101	
    Williams	CS-190	
    Zhang       CS-101	
    Zhang       CS-347
    \end{lstlisting}
        
\section{Query 5}

    \textbf{For all courses with an instructor taught prior to or during 2010, show the course id, the year taught, and the name of the instructor.  (Use the teaches and instructor tables for this one).}
    \begin{lstlisting}
    SELECT name, course_id, year FROM instructor, teaches WHERE year < 2011 AND teaches.ID=instructor.ID


    name	    course_id	year	
    Srinivasan	CS-101   	2009	
    Srinivasan	CS-315   	2010	
    Srinivasan	CS-347   	2009	
    Wu	        FIN-201  	2010	
    Mozart	    MU-199   	2010	
    Einstein	PHY-101    	2009	
    El Said 	HIS-351	    2010
    Katz    	CS-101	    2010
    Katz    	CS-319	    2010
    Crick   	BIO-101	    2009
    Crick   	BIO-301	    2010
    Brandt  	CS-190	    2009
    Brandt  	CS-190	    2009
    Brandt  	CS-319	    2010
    Kim	        EE-181	    2009
    \end{lstlisting}
        
\section{Query 6}

    \textbf{How many unique courses were offered in 2010?  By "unique", I mean unique course ids.   Use the section table for this query.  Call the resulting column NumCourses2010.}
    \begin{lstlisting}
    SELECT DISTINCT course_id as NumCourses2010 FROM section WHERE year=2010
    
    
    NumCourses2010	
    CS-101	
    FIN-201	
    MU-199	
    BIO-301	
    HIS-351	
    CS-319	
    CS-315	
    \end{lstlisting}
        
\section{Query 7}

    \textbf{Make a list of all names (instructors and students).  Your resulting table should have one column.}
    \begin{lstlisting}
    SELECT name FROM instructor UNION SELECT name FROM student
    
    
    name	
    Srinivasan	
    Wu	
    Mozart	
    Einstein	
    El Said	
    Gold	
    Katz	
    Califieri	
    Singh	
    Crick	
    Brandt	
    Kim	
    Zhang	
    Shankar	
    Chavez	
    Peltier	
    Levy	
    Williams	
    Sanchez	
    Snow	
    Brown	
    Aoi	
    Bourikas	
    Tanaka	       
    \end{lstlisting}
        
\section{Query 8}

    \textbf{List the course ID, semester, year and title of each course offered by the Finance department.}
    \begin{lstlisting}
    SELECT s.course_id, semester, year, title FROM section AS s, course AS c WHERE s.course_id=c.course_id AND c.dept_name="Finance"
    
    
    course_id   semester    year    title
    FIN-201	    Spring	    2010	Investment Banking	
    \end{lstlisting}
        
\section{Query 9}

    \textbf{List the names of all students and the ID of their advisor, if they have one.  In mysql, left/right joins implement outer joins that we used in relational algebra.}
    \begin{lstlisting}
    SELECT name, i_ID FROM student LEFT JOIN advisor on student.ID=advisor.s_id
    
    
    name    	i_ID	
    Zhang   	45565	
    Shankar 	10101	
    Brandt  	NULL
    Chavez  	76543	
    Peltier 	22222	
    Levy    	22222	
    Williams    NULL
    Sanchez 	NULL
    Snow    	NULL
    Brown   	45565	
    Aoi 	    98345	
    Bourikas    98345	
    Tanaka  	76766
    \end{lstlisting}
        
\section{Query 10}

    \textbf{For each student who has taken a course, list their name, the name of the student's department and the name of the course that they have taken.}
    \begin{lstlisting}
    SELECT name, course.dept_name, title FROM student, takes, course WHERE student.ID=takes.ID AND takes.course_id=course.course_id
    
    
    name    	dept_name	title	
    Zhang	    Comp. Sci.	Intro. to Computer Science	
    Zhang   	Comp. Sci.	Database System Concepts	
    Shankar 	Comp. Sci.	Intro. to Computer Science	
    Shankar 	Comp. Sci.	Game Design	
    Shankar 	Comp. Sci.	Robotics	
    Shankar 	Comp. Sci.	Database System Concepts	
    Brandt  	History	    World History	
    Chavez  	Finance	    Investment Banking	
    Peltier 	Physics	    Physical Principles	
    Levy    	Comp. Sci.	Intro. to Computer Science	
    Levy    	Comp. Sci.	Intro. to Computer Science	
    Levy    	Comp. Sci.	Image Processing	
    Williams	Comp. Sci.	Intro. to Computer Science	
    Williams	Comp. Sci.	Game Design	
    Sanchez 	Music	    Music Video Production	
    Brown   	Comp. Sci.	Intro. to Computer Science	
    Brown   	Comp. Sci.	Image Processing	
    Aoi 	    Elec. Eng.	Intro. to Digital Systems	
    Bourikas    Comp. Sci.	Intro. to Computer Science	
    Bourikas    Comp. Sci.	Robotics	
    Tanaka  	Biology	    Intro. to Biology	
    Tanaka  	Biology	    Genetics       
    \end{lstlisting}
        
\section{Query 11}

    \textbf{Make a list containing each student's name and the number of unique courses that they have taken in each year.  Your output should have 3 columns, Student Name, Year and Number of Courses.}
    \begin{lstlisting}
        
    \end{lstlisting}
        
\section{Query 12}

    \textbf{Make a list of courses (course ids) that were offered in both Fall 2009 and in Spring 2010.   (This does not mean either/or, but courses that were offered in both semesters).}
    \begin{lstlisting}
    SELECT course_id FROM teaches WHERE semester="Fall" and year=2009 INTERSECT SELECT course_id FROM teaches WHERE semester="Spring" and year=2010
       
       
    course_id
    CS-101
    \end{lstlisting}
        
\section{Query 13}

    \textbf{Show students (just their ID is enough) who have both an A and an C among the courses that they have taken.}
    \begin{lstlisting}
    SELECT ID FROM takes WHERE grade="A" INTERSECT SELECT ID FROM takes WHERE grade="C"


    ID
    12345
    \end{lstlisting}
        
\section{Query 14}

    \textbf{For all courses that start at 8:00AM, change the time to 9:30AM.} \\
    As a table update has no output, to show this query's success, three things will be shown:
    \begin{enumerate}
        \item The SQL query
        \item The courses that initially start at 8:00AM and 9:30AM
        \item The courses that start at 8:00AM and 9:30AM after altering the table
    \end{enumerate}
    The SQL query is below:
    \begin{lstlisting}
    UPDATE time_slot SET start_time="9:30" WHERE start_time="8:00";
    \end{lstlisting}
    The courses that initially start at 8:00AM and 9:30AM are listed below
    \begin{lstlisting}
    SELECT course_id, sec_id, semester, year, start_time FROM section,time_slot WHERE start_time="8:00" OR start_time="9:30"
       
       
    course_id	sec_id	semester	year	start_time	
    CS-101  	1	    Fall	    2009	08:00:00	
    CS-101  	1	    Spring	    2010	08:00:00	
    FIN-201 	1	    Spring	    2010	08:00:00	
    MU-199  	1	    Spring	    2010	08:00:00	
    BIO-101 	1	    Summer	    2009	08:00:00	
    BIO-301 	1	    Summer	    2010	08:00:00	
    HIS-351 	1	    Spring	    2010	08:00:00	
    CS-190  	1	    Spring	    2009	08:00:00	
    CS-190  	2	    Spring	    2009	08:00:00	
    CS-319  	2	    Spring	    2010	08:00:00	
    CS-347  	1	    Fall  	    2009	08:00:00	
    EE-181  	1	    Spring    	2009	08:00:00	
    CS-319  	1	    Spring    	2010	08:00:00	
    PHY-101 	1	    Fall  	    2009	08:00:00	
    CS-315  	1	    Spring    	2010	08:00:00	
    CS-101  	1	    Fall  	    2009	08:00:00	
    CS-101  	1	    Spring    	2010	08:00:00	
    FIN-201 	1	    Spring    	2010	08:00:00	
    MU-199  	1	    Spring    	2010	08:00:00	
    BIO-101 	1	    Summer    	2009	08:00:00	
    BIO-301 	1  	    Summer      2010	08:00:00	
    HIS-351 	1  	    Spring      2010	08:00:00	
    CS-190  	1  	    Spring      2009	08:00:00	
    CS-190  	2  	    Spring      2009	08:00:00	
    CS-319  	2  	    Spring      2010	08:00:00	
    CS-347  	1  	    Fall        2009	08:00:00	
    EE-181  	1  	    Spring      2009	08:00:00	
    CS-319  	1  	    Spring      2010	08:00:00	
    PHY-101 	1  	    Fall        2009	08:00:00	
    CS-315  	1  	    Spring      2010	08:00:00	
    CS-101  	1       Fall    	2009	08:00:00	
    CS-101  	1       Spring  	2010	08:00:00	
    FIN-201 	1       Spring  	2010	08:00:00	
    MU-199  	1       Spring  	2010	08:00:00	
    BIO-101 	1       Summer  	2009	08:00:00	
    BIO-301 	1       Summer  	2010	08:00:00	
    HIS-351 	1       Spring  	2010	08:00:00	
    CS-190  	1       Spring  	2009	08:00:00	
    CS-190  	2       Spring  	2009	08:00:00	
    CS-319  	2       Spring  	2010	08:00:00	
    CS-347  	1      	Fall	    2009	08:00:00	
    EE-181  	1      	Spring	    2009	08:00:00	
    CS-319  	1      	Spring	    2010	08:00:00	
    PHY-101 	1      	Fall	    2009	08:00:00	
    CS-315  	1      	Spring	    2010	08:00:00
    \end{lstlisting}
    The courses that start at 8:00AM and 9:30AM after altering the table are below:
    \begin{lstlisting}
    SELECT course_id, sec_id, semester, year, start_time FROM section,time_slot WHERE start_time="8:00" OR start_time="9:30"
    
    
    course_id	sec_id	semester	year	start_time	
    CS-101  	1	    Fall 	    2009	09:30:00	
    CS-101  	1	    Spring   	2010	09:30:00	
    FIN-201 	1	    Spring   	2010	09:30:00	
    MU-199  	1	    Spring   	2010	09:30:00	
    BIO-101 	1	    Summer   	2009	09:30:00	
    BIO-301 	1	    Summer   	2010	09:30:00	
    HIS-351 	1	    Spring   	2010	09:30:00	
    CS-190  	1	    Spring   	2009	09:30:00	
    CS-190  	2	    Spring   	2009	09:30:00	
    CS-319  	2	    Spring   	2010	09:30:00	
    CS-347  	1	    Fall 	    2009	09:30:00	
    EE-181  	1	    Spring   	2009	09:30:00	
    CS-319  	1	    Spring   	2010	09:30:00	
    PHY-101 	1	    Fall 	    2009	09:30:00	
    CS-315  	1	    Spring   	2010	09:30:00	
    CS-101  	1	    Fall 	    2009	09:30:00	
    CS-101  	1	    Spring   	2010	09:30:00	
    FIN-201 	1	    Spring   	2010	09:30:00	
    MU-199  	1	    Spring   	2010	09:30:00	
    BIO-101 	1	    Summer   	2009	09:30:00	
    BIO-301 	1	    Summer   	2010	09:30:00	
    HIS-351 	1	    Spring   	2010	09:30:00	
    CS-190  	1	    Spring   	2009	09:30:00	
    CS-190  	2	    Spring   	2009	09:30:00	
    CS-319  	2	    Spring   	2010	09:30:00	
    CS-347  	1	    Fall 	    2009	09:30:00	
    EE-181  	1	    Spring   	2009	09:30:00	
    CS-319  	1	    Spring   	2010	09:30:00	
    PHY-101 	1	    Fall 	    2009	09:30:00	
    CS-315  	1	    Spring   	2010	09:30:00	
    CS-101  	1	    Fall 	    2009	09:30:00	
    CS-101  	1	    Spring   	2010	09:30:00	
    FIN-201 	1	    Spring   	2010	09:30:00	
    MU-199  	1	    Spring   	2010	09:30:00	
    BIO-101 	1	    Summer   	2009	09:30:00	
    BIO-301 	1	    Summer   	2010	09:30:00	
    HIS-351 	1	    Spring   	2010	09:30:00	
    CS-190  	1	    Spring   	2009	09:30:00	
    CS-190  	2	    Spring   	2009	09:30:00	
    CS-319  	2	    Spring   	2010	09:30:00	
    CS-347  	1      	Fall	    2009	09:30:00	
    EE-181  	1      	Spring	    2009	09:30:00	
    CS-319  	1      	Spring	    2010	09:30:00	
    PHY-101 	1      	Fall	    2009	09:30:00	
    CS-315  	1      	Spring	    2010	09:30:00       
    \end{lstlisting}
        
\section{Query 15}

    \textbf{Delete the Comp. Sci. department.}
    As a table delete has no output, to show this query's success, five things will be shown:
    \begin{enumerate}
        \item The SQL query
        \item The initial departments
        \item The initial courses
        \item The departments after altering the table
        \item The courses after altering the table
    \end{enumerate}
    The reason that the courses are not also deleted, but rather, their department is set to null, is because the university may offer this course again in the future but in a different department. At Queen's, we saw this with APSC 143 becoming MNTC 313 from Winter 2020 to Summer 2020.
    The SQL query is below:
    \begin{lstlisting}
    DELETE FROM department WHERE dept_name="Comp. Sci.";
    \end{lstlisting}
    The initial departments are below
    \begin{lstlisting}
    dept_name	building	budget	
    Biology	    Watson	    90000.00
    Comp. Sci.	Taylor	    100000.00	
    Elec. Eng.	Taylor	    85000.00
    Finance	    Painter	    120000.00
    History	    Painter	    50000.00
    Music	    Packard	    80000.00
    Physics	    Watson	    70000.00
    \end{lstlisting}
    The initial courses are below
    \begin{lstlisting}
    course_id	title	                    dept_name	  credits	
    BIO-101	    Intro. to Biology	        Biology	      4
    BIO-301	    Genetics	                Biology	      4
    BIO-399	    Computational Biology	    Biology       3
    CS-101	    Intro. to Computer Science	Comp. Sci.	  4
    CS-190	    Game Design	                Comp. Sci.    4
    CS-315	    Robotics	                Comp. Sci.    3
    CS-319	    Image Processing	        Comp. Sci.    3
    CS-347	    Database System Concepts	Comp. Sci.    3
    EE-181	    Intro. to Digital Systems	Elec. Eng.    3
    FIN-201	    Investment Banking	        Finance	      3
    HIS-351	    World History	            History	      3
    MU-199	    Music Video Production	    Music	      3
    PHY-101	    Physical Principles	        Physics	      4
    \end{lstlisting}
    The department list after running the query is below:
    \begin{lstlisting}
    dept_name	building	budget	
    Biology	    Watson	    90000.00
    Elec. Eng.	Taylor	    85000.00
    Finance	    Painter     120000.00
    History	    Painter     50000.00
    Music	    Packard     80000.00
    Physics	    Watson	    70000.00
    \end{lstlisting}
    The course list after running the query is below:
    \begin{lstlisting}
    course_id	title	                    dept_name	credits	
    BIO-101	    Intro. to Biology	        Biology	    4
    BIO-301	    Genetics	                Biology	    4
    BIO-399	    Computational Biology	    Biology	    3
    CS-101	    Intro. to Computer Science  NULL        4	
    CS-190	    Game Design                 NULL        4	
    CS-315	    Robotics                    NULL        3	
    CS-319	    Image Processing            NULL        3	
    CS-347	    Database System Concepts    NULL        3
    EE-181	    Intro. to Digital Systems	Elec. Eng.	3	
    FIN-201	    Investment Banking	        Finance	    3
    HIS-351	    World History	            History	    3
    MU-199	    Music Video Production	    Music	    3
    PHY-101	    Physical Principles	        Physics	    4      
    \end{lstlisting}

\end{document}